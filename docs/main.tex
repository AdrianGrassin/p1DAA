\documentclass[11pt]{report}

% Paquetes y configuraciones adicionales
\usepackage{graphicx}
\usepackage[export]{adjustbox}
\usepackage{caption}
\usepackage{float}
\usepackage{titlesec}
\usepackage{geometry}
\usepackage[hidelinks]{hyperref}
\usepackage{titling}
\usepackage{titlesec}
\usepackage{parskip}
\usepackage{wasysym}
\usepackage{tikzsymbols}
\usepackage{fancyvrb}
\usepackage{xurl}
\usepackage{hyperref}
\usepackage{subcaption}

\usepackage{listings}
\usepackage{xcolor}

\usepackage[spanish]{babel}

\newcommand{\subtitle}[1]{
  \posttitle{
    \par\end{center}
    \begin{center}\large#1\end{center}
    \vskip0.5em}
}

% Configura los márgenes
\geometry{
  left=2cm,   % Ajusta este valor al margen izquierdo deseado
  right=2cm,  % Ajusta este valor al margen derecho deseado
  top=3cm,
  bottom=3cm,
}

% Configuración de los títulos de las secciones
\titlespacing{\section}{0pt}{\parskip}{\parskip}
\titlespacing{\subsection}{0pt}{\parskip}{\parskip}
\titlespacing{\subsubsection}{0pt}{\parskip}{\parskip}

% Redefinir el formato de los capítulos y añadir un punto después del número
\makeatletter
\renewcommand{\@makechapterhead}[1]{%
  \vspace*{0\p@} % Ajusta este valor para el espaciado deseado antes del título del capítulo
  {\parindent \z@ \raggedright \normalfont
    \ifnum \c@secnumdepth >\m@ne
        \huge\bfseries \thechapter.\ % Añade un punto después del número
    \fi
    \interlinepenalty\@M
    #1\par\nobreak
    \vspace{10pt} % Ajusta este valor para el espacio deseado después del título del capítulo
  }}
\makeatother

% Configura para que cada \chapter no comience en una pagina nueva
\makeatletter
\renewcommand\chapter{\@startsection{chapter}{0}{\z@}%
    {-3.5ex \@plus -1ex \@minus -.2ex}%
    {2.3ex \@plus.2ex}%
    {\normalfont\Large\bfseries}}
\makeatother

% Configurar los colores para el código
\definecolor{codegreen}{rgb}{0,0.6,0}
\definecolor{codegray}{rgb}{0.5,0.5,0.5}
\definecolor{codepurple}{rgb}{0.58,0,0.82}
\definecolor{backcolour}{rgb}{0.95,0.95,0.92}

% Configurar el estilo para el código
\lstdefinestyle{mystyle}{
  backgroundcolor=\color{backcolour},   
  commentstyle=\color{codegreen},
  keywordstyle=\color{magenta},
  numberstyle=\tiny\color{codegray},
  stringstyle=\color{codepurple},
  basicstyle=\ttfamily\footnotesize,
  breakatwhitespace=false,         
  breaklines=true,                 
  captionpos=b,                    
  keepspaces=true,                 
  numbers=left,                    
  numbersep=5pt,                  
  showspaces=false,                
  showstringspaces=false,
  showtabs=false,                  
  tabsize=2
}

\begin{document}

% Portada del informe
\title{Primera práctica}
\subtitle{Diseño y Análisis de Algoritmos}
\author{Adrián Grassin Luis (alu01013494809@ull.edu.es)}
\date{\today}

\maketitle

\pagestyle{empty} 

\tableofcontents

% Nueva página
\cleardoublepage

\pagestyle{plain} % Vuelve a activar la numeración de página
\setcounter{page}{1} % Reinicia el contador de página a 1

% Secciones del informe
\chapter{Principios Solid}
Los principios SOLID son un conjunto de reglas que nos ayudan a diseñar 
una buena arquitectura de software. Estos principios fueron acuñados por 
Robert C. Martin (Uncle Bob) a principios de la década del 2000.
\begin{itemize}
      \item \textbf{S}ingle Responsability Principle
      \item \textbf{O}pen/Closed Principle
      \item \textbf{L}iskov Substitution Principle
      \item \textbf{I}nterface Segregation Principle
      \item \textbf{D}ependency Inversion Principle
\end{itemize}
\paragraph{Single Responsability Principle\newline} 
El objetivo del principio de responsabilidad única se basa en que cada 
clase módulo sea responsable de una sola parte de la funcionalidad proporcionada 
por el software y esta responsabilidad debe estar encapsulada en su totalidad 
por la clase. Es decir, que si una clase tiene más de una responsabilidad, esta se
debería subdividir en clases más pequeñas, cada una con su única responsabilidad haciendo que 
la lógica se encuentre separada y sea más fácil de mantener.
\chapter{Patrón de estrategia}
El patrón de estrategia es un patrón de diseño de software que permite seleccionar un algoritmo
en tiempo de ejecución. Esto se logra encapsulando cada algoritmo (estrategia) en una clase separada normalmente
siendo esta una clase hija de una interfaz o una clase abstracta de la cual se heredan los métodos que implementan el algoritmo.

Un ejemplo de este patrón podría verse en una práctica de la asignatura de \textit{Algoritmos y 
Estructuras de Datos Avanzadas} de hace unos años en la que
se implementaba un \textit{juego de la vida de Conway} con distintas reglas de supervivencia, nacimiento y varios tipos de mundo. 
Al ejecutar el programa se podía elegir entre distintas reglas de supervivencia y nacimiento y distintos tipos de mundo.
mediante el uso de punteros en el código a instancias hijas de la clase que implementaba la interfaz de reglas de supervivencia y nacimiento. 
Un ejemplo concreto podría ser el cálculo de las celulas adyacentes de cada célula, pudiendo elegir si
el algoritmo de cálculo de adyacencia que funcionase con un mundo ``sin bordes'' o con un mundo finito, cambiando así
la estrategia en la que se comprobaba el entorno de la celula.

\cleardoublepage

\section*{Referencias}
\begin{enumerate}
      \item \url{https://www.freecodecamp.org/espanol/news/los-principios-solid-explicados-en-espanol/}
      \item \url{ https://www.digitalocean.com/community/conceptual-articles/s-o-l-i-d-the-first-five-principles-of-object-oriented-design}
      \item \url{https://www.freecodecamp.org/news/a-beginners-guide-to-the-strategy-design-pattern/}
      \item \url{https://refactoring.guru/es/design-patterns/strategy}
\end{enumerate}


\end{document}
